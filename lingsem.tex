% -----------------------------------------------------------------------------------------------------------------
% lingsem: MACROS AND EXAMPLES OF WRITING SEMANTICS IN LaTeX 
% -----------------------------------------------------------------------------------------------------------------
	
	% -------------------------------------------------------------------------------------------------------------
	% PACKAGES NEEDED FOR WRITING LINGUISTICS SEMANTICS DENOTATION MACROS AND EXPLANATIONS
	% -------------------------------------------------------------------------------------------------------------
	
	\usepackage{stmaryrd} % used for the denotation brackets ('[[' and ']]'), which are part of this package and are written so as to be delimiters
	\usepackage{amsmath} % used to provide the \text command so that the text inside the denotation brackets is _not_ italic 
	\usepackage{ragged2e} % used to provide the \RaggedRight command, which wraps text with both a ragged-right margin and hyphenation
	\usepackage{varwidth} % used to set the width of wrapped text to its natural width, lest the rightmost denotation bracket be offset based on the specified width

	% -------------------------------------------------------------------------------------------------------------
	% MACRO FOR THE INTERPRETATION FUNCTION
	% -------------------------------------------------------------------------------------------------------------
	
	\newcommand{\interp}[2][]{
	\(
		\left\llbracket\,\text{#2}\,\right\rrbracket^{#1}
	\)
	}

	% -------------------------------------------------------------------------------------------------------------
	% MACRO FOR WRAPPING LONG TEXT INSIDE DENOTATION
	% -------------------------------------------------------------------------------------------------------------
	
	\newcommand{\wraptext}[2][1in]{\begin{varwidth}{#1}{\RaggedRight#2}\end{varwidth}}
	
	% -------------------------------------------------------------------------------------------------------------
	% MACRO FOR LAMBDA EXPRESSIONS
	% -------------------------------------------------------------------------------------------------------------

	\newcommand{\lam}[2][]{$\lambda {#2}_{#1}$.}
	
	% -------------------------------------------------------------------------------------------------------------
	% MACRO FOR EXPLICIT LAMBDA EXPRESSIONS
	% -------------------------------------------------------------------------------------------------------------

	\newcommand{\lamexp}[2]{$\lambda {#1} \in D_{#2}$.}

	% -------------------------------------------------------------------------------------------------------------
	% MACRO FOR DENOTATION BRACKETS, TO BE USED WITH \WRAPTEXT
	% -------------------------------------------------------------------------------------------------------------

	\newcommand{\den}[1]{
	\(
		\left[\,\text{#1}\,\right]
	\)
	}

	% -------------------------------------------------------------------------------------------------------------
	% MACRO FOR ARGUMENT PARENTHESES, TO BE USED WITH \WRAPTEXT
	% -------------------------------------------------------------------------------------------------------------

	\newcommand{\argum}[1]{
	\(
		\left(\,\text{#1}\,\right)
	\)
	}

% -----------------------------------------------------------------------------------------------------------------
% END SEMANTICS MACROS
% -----------------------------------------------------------------------------------------------------------------